%*****************************************************
%	APPENDIX
%*****************************************************
\appendix
%\renewcommand*{\thesection}{\alph{section}}
\chapter{Appendix}
\label{ch:appendix}
%-----------------------------------------------------
% Here are some useful LaTEX examples.
% - Block diagrams 
% -	Equations
% - Figures
% - Code
% - Circuit diagrams
% Please note that there often many ways of entering
% this type of information. 
%====================================================
\section{Repository for the code}
\label{sec:appendix.python_solution}

Follow the read me file in the GitHub repository to replicate the work done in this thesis: \url{https://github.com/dikokob/DikokoMScEng}. The read me file also includes the process of setting up the turtle bot.

\section{Raw data-sets}
\label{sec:appendix.rawdata}

Raw data-sets used in Chapters \ref{ch:experimental} and \ref{sec:map-merging} can be found in: \url{https://drive.google.com/drive/folders/1c\_2-T8FcrSmE0VDyHw20-jRLHKRuAvpE?usp=sharing}

\section{Hector mapping parameters}
\label{sec:appendix.hector-mapping}


\begin{center}
\begin{tabular}{ | m{2cm} | m{3cm} | m{10cm} |  } 
\hline
\textbf{Parameter} & \textbf{Value} & \textbf{Description}\\ 
\hline
\hline
map\_resolution & Varies based on experiment & The map resolution m/cell. This is the length of a grid cell edge.\\ 
\hline
map\_size & 2048 & The size (number of cells per axis) of the map. The map is square and has (map\_size * map\_size) grid cells.\\ 
\hline
map\_start\_x & 0.5(default) & Location of the origin (0.0, 1.0) of the /map frame on the x axis relative to the grid map. 0.5 is in the middle.\\ 
\hline
map\_start\_y & 0.5(default) & Location of the origin (0.0, 1.0) of the /map frame on the y axis relative to the grid map. 0.5 is in the middle.\\ 
\hline
map\_update\_distance\_thresh  & 0.4 (default) & Threshold for performing map updates (m). The platform has to travel this far in meters or experience an angular change as described by the map\_update\_angle\_thresh parameter since the last update before a map update happens. \\ 
\hline
map\_update\_angle\_thresh & 0.9(default) & Threshold for performing map updates (rad). The platform has to experience an angular change as described by this parameter of travel as far as specified by the map\_update\_distance\_thresh parameter since the last update before a map update happens.  \\ 
\hline
map\_pub\_period & 2.0 & The map publish period (s).  \\ 
\hline
map\_multi\_res\_levels & 3(default) & The number of map multi-resolution grid levels.  \\ 
\hline
update\_factor\_free  & 0.4(default) &  The map update modifier for updates of free cells in the range (0.0, 1.0). A value of 0.5 means no change.\\ 
\hline
update\_factor\_occupied   & 0.9(default) &  The map update modifier for updates of occupied cells in the range (0.0, 1.0). A value of 0.5 means no change.\\ 
\hline
laser\_min\_dist & 0.1 & The minimum distance (m) for laser scan endpoints to be used by the system. Scan endpoints closer than this value are ignored.\\ 
\hline
laser\_max\_dist & 30(default) & The maximum distance (m) for laser scan endpoints to be used by the system. Scan endpoints farther away than this value are ignored. \\ 
\hline
laser\_z\_min\_value & -1.0(default) & The minimum height (m) relative to the laser scanner frame for laser scan endpoints to be used by the system. Scan endpoints lower than this value are ignored. \\ 
\hline
laser\_z\_max\_value & 1.0(default) & The maximum height (m) relative to the laser scanner frame for laser scan endpoints to be used by the system. Scan endpoints higher than this value are ignored.\\ 
\hline
\end{tabular}
\end{center}

