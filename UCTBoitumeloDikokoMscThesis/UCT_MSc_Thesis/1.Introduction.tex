% !TeX spellcheck = en_GB
%****************************************************
%	CHAPTER 1 - INTRODUCTION
%****************************************************
\chapter{Introduction}
\label{ch:ch1}
%====================================================
\section{A brief background to the study}
\label{sec:ch1.section1}
%----------------------------------------------------
A key feature of robot autonomy is the ability of a platform to both sense its environment and situate itself within it \cite{Siegwart2004}. Therefore, while exploring an unknown environment, an autonomous mobile robot must build a local map and position itself within the map using available sensor data (such as LIDAR, cameras and inertial measurement units (IMU)). This process is referred to as Simultaneous Localisation and Mapping (SLAM) \cite{Bailey2006b, Bailey2006a}.

Multi-Robot SLAM (MRSLAM) is a far more robust and efficient strategy when exploring large areas, as computational burdens are shared across many robots.  Multiple robots reduce the time needed to map the same area by exploring different parts of the environment in parallel. The robot team is robust as a decentralised formation as the failure of one robot does not necessarily halt the mission \cite{Leung2010}, and for a mapping mission to be completely decentralised, each robot needs to operate independently, and no robot should be of higher importance than any other robot in the mapping task. Furthermore, local maps generated by the robots can be shared amongst the platforms and merged to be used for additional mission tasks such as navigation and path planning.

Even though using MRSLAM can be advantageous, the following considerations arise when merging a map:
\begin{itemize}
    \item Unknown relative robot poses and the uncertainty associated with their local and relative positions in a global map.
    \item Varying map representation, resolution and orientation across robots.
    \item incorporating both location and mapping uncertainty into the final map fusion \cite{Saeedi}.
\end{itemize}
 
Merging existing maps from multiple robots is the process of using raw map data from multiple robots to produce a final global map and as such can result in various map representations across the robot platforms. There are two approaches: one approach takes raw data from all the robots, produces a global map centrally and then shares the resulting map with all the robots (this may need to be transformed into a representation that the local robot can use), and the second approach robots share raw data among each other, and each robot produces their own global map \cite{Saeedi}. 

Previous MRSLAM approaches have often assumed robots know their location in a global map using measurements from Global Positioning Systems (GPS) \cite{Sukkarieh2006,Kim2007a}. With this assumption, robots have a straightforward objective to explore as much of the map as possible, and it further allows for more detailed map merging as the ground truth is always known. However, GPS is not always available, especially in indoor environments. Work done by \cite{Sheng2006}, \cite{Matari2004} addresses the lack of GPS data, by relying on rendezvous positions in the environment to verify the relative robot pose and to initialise a map merging event. The drawback of this is if robots fail to reach the rendezvous position, a merging event may never occur. 

Other studies approach map merging as an optimisation problem \cite{Carpin2005a,XinMa2008,Carpin2008}, by performing a transformation that optimises map overlap. These approaches promise an optimal solution when the number of iterations approaches infinity, although optimality is not always guaranteed in real-time applications. Saeedi et al. tackle the map merging problem as a particular case of image registration \cite{Saeedi2014c, Saeedi2012a}. Even though this method has proven to be computationally superior to the others, they assume that the initial maps are of the same scale and generated at the same resolution. This approach is not robust when multi-robots generate maps of different scales and resolutions or when higher resolution is needed in a particular map region, and this, therefore,  needs to be considered. Multiple resolution maps are ideal in scenarios such as a mine, as mines are usually large environments, and some areas may include more information than others.  For areas where the environment has many details, a fine resolution map can be built; and for areas, with fewer details, a coarse resolution map can be generated. Thus, it is necessary to consider merging multiple grid maps at different scales and resolutions.

%However, grid maps could be built at different resolutions by various robots in multi-robot systems. Suppose, for example; it is necessary to build a grid map for an abandoned mine.35 As the mine is a large environment, multi-robot systems could be adopted to build the grid map. In general, some areas of the mine may include more details than others. For compactness purposes, the map could be built at a fine resolution in the area with more details and a coarse resolution in the area with fewer details. Thus, it is necessary to consider the problem of merging grid maps at different resolutions.

%====================================================
\section{Problem statement}
\label{sec:ch1.section2}
%----------------------------------------------------
This thesis was motivated by the need to map known and unknown structured environments using multiple mobile robot platforms. Each mobile robot needs to produce a local map, retrieve a previous map, and update the current map. Maps provided by multiple robots are then merged on-board each robot to produce a global map locally. In general, some areas of the environment may include more details than others. For compactness purposes, the map could be built at a fine resolution in the area with more details and a coarse resolution in the area with fewer details. Thus, maps generated at various resolutions need also to be handled by the merging process.


%====================================================
\section{Objective of research}
\label{sec:ch1.section3}
%----------------------------------------------------
The research aim in this study is to investigate and produce an implementation of Multiple Mobile Robot SLAM for Collaborative Mapping and Exploration. The project was focused on two main components of this process: (a) multi-session mapping and (b) map merging.

In this thesis, multi-session mapping uses an existing local map to update and expand on a previous map of the environment. This means a robot will either use a map previously created by another robot or itself, to update and expand the current local map. The robot needs to either localise within the previously mapped region or match features in its local map to that region.

The map merging algorithm was based on a multi-session implementation and used a 2D occupancy grid map to represent the mapped structured environments. Application of the map merging algorithm was conducted in a distributed manner across robot platforms to deal with potential communication loss between the mobile robots in the environment. Also, since the relative pose geometric transformation between the robots is unknown, improved pose geometric transformation estimation techniques are considered.

Additionally, this study addressed problems of: (a) unknown re-entry into a previously mapped area and (b) map merging of maps created by independent robots with unknown initial poses, and unknown relative pose transformations between the robots.


%====================================================
\section{Thesis scope and contributions}
\label{sec:ch1.section4}
%----------------------------------------------------
In this thesis, an occupancy grid-based multi-session mapping and the merging algorithm is developed. In this approach robots individually build maps using HectorSLAM to generate initial local occupancy grid maps of the environment. When robots can communicate, local maps are then shared with other robots. An image registration technique, Scale-invariant feature transform (SIFT), is used to identify common features between the maps before merging takes place \cite{Kohlbrecher2011a}. It can handle both mappings; mapping a new unknown environment and merging a current local map to a previously mapped region. The multi-session map merging algorithm includes rules that address multi-session mapping and merging of maps generated by other mobile robot platforms with unknown initial position and relative pose transformations. Additionally, it can identify overlap or the lack of overlap between multiple maps and is run in a decentralised manner on each robot in the system. 

The developed algorithm is implemented on ROS (Robot Operating System) using python and validated in simulation (using Gazebo Simulator) and real-world environments. In the real-world environments, raw data from the Ray and Maria Stata Center \footnote{https://projects.csail.mit.edu/stata/} and raw data produced for this thesis using a Kobuki platform were used. The Kobuki platform and the platform used in the Ray and Maria Stata Center dataset were both equipped with a two-dimensional Hokuyo UTM-30LX LIDAR scanner in an indoor structured environment.

This approach us designed to work with data generated from a two-dimensional laser scanner and is not evaluated for a three-dimensional laser scanner. To simulate the algorithm's real-time running, the datasets were stored as rosbag files (a ROS data store format) then played back as though multiple robots were recording data as different locations in the map. Additionally, all maps were represented using occupancy grid maps and were shared through ROS communication, then converted into image files before map merging events on each robot. Transmission of map data between robots and storage constraints of maps on-board each robot were not evaluated in this study. 

%====================================================
\section{Plan of development}
\label{sec:ch1.section5}
%----------------------------------------------------
This thesis is organised into six chapters, bibliography and appendices, each referring to the project phases. Firstly this introductory chapter introduces the context and motivation for this study. Chapter \ref{ch:ch2} provides background to the SLAM problem as a basis for some of the study choices; furthermore, the chapter reviews the advantages and disadvantages of existing multiple-robot SLAM algorithms. Chapter \ref{sec:map-merging} address map merging, which includes parameter tuning for an image registration implementation. Chapter \ref{ch:implementation} addresses the implementation of multiple-robot SLAM and several important steps to achieve this using ROS. In this chapter, more details of the basic ROS modules are given for the multiple-robots used in this work.  In Chapter \ref{ch:experimental}, the experimental validation results are presented for both the simulated and real-world environments. Advantages and disadvantages of the approach are also highlighted and discussed. To conclude, Chapter \ref{ch:conclusion} sums up the work, providing conclusions and suggestions for future work for this research.
%****************************************************
% END
%****************************************************
